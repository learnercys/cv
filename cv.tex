%%%%%%%%%%%%%%%%%%%%%%%%%%%%%%%%%%%%%%%%%
% Friggeri Resume/CV
% XeLaTeX Template
% Version 1.0 (5/5/13)
%
% This template has been downloaded from:
% http://www.LaTeXTemplates.com
%
% Original author:
% Adrien Friggeri (adrien@friggeri.net)
% https://github.com/afriggeri/CV
%
% CC BY-NC-SA 3.0 (http://creativecommons.org/licenses/by-nc-sa/3.0/)
%
% License:
% Important notes:
% This template needs to be compiled with XeLaTeX and the bibliography, if used,
% needs to be compiled with biber rather than bibtex.
%
%%%%%%%%%%%%%%%%%%%%%%%%%%%%%%%%%%%%%%%%%

\documentclass[]{friggeri-cv} % Add 'print' as an option into the square bracket to remove colors from this template for printing

\addbibresource{bibliography.bib} % Specify the bibliography file to include publications

\begin{document}

\header{Carlos}{Hernández}{Web Developer} % Your name and current job title/field

%----------------------------------------------------------------------------------------
%	SIDEBAR SECTION
%----------------------------------------------------------------------------------------

\begin{aside} % In the aside, each new line forces a line break
\section{Contacto}
Ciudad de Guatemala
 (502) 4276-6154
~
\href{mailto:vgh.carlos@gmail.com}{vgh.carlos@gmail.com}
\href{https://www.linkedin.com/pub/carlos-hern\%C3\%A1ndez/87/1b1/157}{linkedin}
%\href{http://learnercys.wordpress.com/}{blog}
\href{https://github.com/learnercys}{github}
\section{Idiomas}
Español
Inglés Intermedio 
\end{aside}

%----------------------------------------------------------------------------------------
%	WORK EXPERIENCE SECTION
%----------------------------------------------------------------------------------------

\section{Experiencia}
%------------------------------------------------
\begin{entrylist}
\entry
	{2014-presente}
	{Kipo Inc.}
	{Ciudad de Guatemala}
	{\emph{Front-end Developer} \\
	Coordinar y calendarizar tareas en el desarrollo del Frontend, crear la arquitectura y difinir las tecnologías a usar por proyecto.
	\begin{itemize}
		\item Desarrollo de soluciones para la geolocalización en tiempo real usando Google Maps y AngularJS.
		\item Construcción de Datos basados en las librerías D3js y Google Charts.
		\item Integración con APIs basadas en servicios RESTful.
	\end{itemize}
	
	}
	
\entry
	{2013--2014}
	{CLARO}
	{Ciudad de Guatemala}
	{\emph{Analista de Sistemas} \\
	Organizar el desarrollo y soporte para los proyectos pertinentes en el Área Móvil.
	\begin{itemize}
		\item Desarrollo de herramientas a nivel regional para el área móvil. 
		\item Manejo de nuevos requerimientos y soporte de sistemas actuales.
		\item Administración de servidores y bases de datos.
		\item Soporte de herramientas.
	\end{itemize}
	
	}

\end{entrylist}

%----------------------------------------------------------------------------------------
%	EDUCATION SECTION
%----------------------------------------------------------------------------------------

\section{Educación}

\begin{entrylist}
%------------------------------------------------
\entry
{2011--presente}
{Ingeniería en Ciencias y Sistemas}
{Universidad de San Carlos de Guatemala, Ciudad de Guatemala	}
{}
%------------------------------------------------

\end{entrylist}


%----------------------------------------------------------------------------------------
%   PROGRAMMING SECTION
%----------------------------------------------------------------------------------------
\section{Cualificaciones}
\textbf{Frontend Developer}
\begin{itemize}
	\item HTML5, Canvas, SVG y Jade.
	\item CSS3, CSS Preprocessors, Bootstrap, Bootstrap Grid System, Boostrap UI, Material Design y Media Queries.
	\item JavaScript, Prototype, DOM Manipulation, jQuery, CoffeeScript, AngularJS, Polymer y NodeJS.
	\item Socket.io.
	\item Google Tools y APIs: Google maps, Analytics y Google Charts.
	\item Grunt y Gulp.

\end{itemize}

\textbf{Backend Developer}

\begin{itemize}
	\item PHP y Laravel.
	\item Python y Django.
	\item .NET.
	\item Servicios RESTful.
	
\end{itemize}

\textbf{Database Adminitrator}
\begin{itemize}
	\item Microsoft SQL Server.
	\item MySQL.
	\item PL/SQL.
\end{itemize}

\textbf{General}
\begin{itemize}
	\item Sistema de Control de Versiones - Git.
	\item Sistemas Operativos: Windows, Linux y iOS.
\end{itemize}



%----------------------------------------------------------------------------------------
%	COMMUNICATION SKILLS SECTION
%----------------------------------------------------------------------------------------


%----------------------------------------------------------------------------------------
%	INTERESTS SECTION
%----------------------------------------------------------------------------------------

\section{intereses}

\textbf{Profesional:} Arquitectura de Software, Inteligencia Artificial, Seguridad y Auditoría Informática. Entusiasta de las nuevas tecnologías de desarrollo. 

\textbf{Personal:} Internet, leer, música y mi familia.

%----------------------------------------------------------------------------------------
%	REFERENCE SECTION
%----------------------------------------------------------------------------------------
\section{Referencias}

\begin{itemize}
	\item \noindent \textbf{Axel Mayorga Ruballos},  Software Architect en Kipo Inc. \hfill \hfill (502) 4216-9351
	\item \noindent \textbf{Kevin Eduardo López}, Software Engineer en Viscosity North America\hfill \hfill (502) 3036-2000
\end{itemize}

%----------------------------------------------------------------------------------------
%	PUBLICATIONS SECTION
%----------------------------------------------------------------------------------------
%----------------------------------------------------------------------------------------

\end{document}
